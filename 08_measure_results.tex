\begin{itemize}
    \item Прилагается \hyperlink{table2}{Таблица 2} в Приложении 2
\end{itemize}

\text{Внутреннее сопротивление источника тока по МНК:} \\
    
    $r = \frac{\sum\limits_{i=1}^{n}(I_i - \overline{I})(U_i - \overline{U}))}{\sum\limits_{i=1}^{n}(I_i - \overline{I})^2} = -0,67635 \ \frac{\text{В}}{\text{мА}}$

    \smallvspace

    \text{ЭДС источника по МНК:} \\
    
    $\mathcal{E} = \overline{U} + \overline{I}|r| = 10,55377 \ \text{В}$

    \newpage

    \text{Значение тока, при котором достигается максимум значения полезной мощности:}

    $I^* = 7.51 \ \text{мА  - по графику } P_R = P_R(I)$

    $I^* = 7.80 \ \text{мА  - по графику } \eta = \eta(I)$

    \smallvspace

    \text{Максимальная полезная мощность:}

    $P_{R max} = 41,1548 \ \text{мВт - по графику } P_R = P_R(I)$

    \smallvspace

    \text{Сопротивление, соответствующее режиму согласования нагрузки и источника}

    $R_{\text{согл}} = \frac{P_{R max}}{(I^*)^2} = 0,7297 \ \frac{\text{В}}{\text{мА}}$