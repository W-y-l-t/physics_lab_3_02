После проведения эксперимента была сформирована выборка, на основе которой были вычислены необходимые косвенные параметры мощностей: полезной, потерь и полной, а также коэффициент полезного действия (КПД) источника. Были построены графики, показывающие зависимости мощностей и КПД от силы тока в цепи. С использованием метода наименьших квадратов были определены значения электродвижущей силы (ЭДС) и внутреннего сопротивления, а также их погрешности. После расчёта силы тока, при которой достигается максимальная полезная мощность, было подтверждено, что в этом случае КПД источника составляет примерно 50\%. Также было найдено значение сопротивления, при котором происходит согласование с источником тока, и установлено, что оно практически совпадает с внутренним сопротивлением источника.